\section{Integration}

\subsection{Trapez}
\begin{itemize}
	\item Einfach \\
	$T(f) = \frac{b-a}{2} (f(a) + f(b))$
	
	\item Zusammengesetzt \\
	$T^{(n)}(f) = h (\frac{1}{2} f(x_0) + f(x_1) + \cdots + f(x_{n-1}) + \frac{1}{2} f(x_n))$
	
	\item Fehler \\
	$\bigg|\int_a^b (f(x)-p(x)) dx \bigg| \leq \frac{M_2(b-a)^3}{12}$

	\item $M_{(n)} = max_{[a,b]}|f^{(n)}(x)|$
	
\end{itemize}

\subsection{Simpson}
\begin{itemize}
	\item Einfach \\
	$S(f) = \frac{h}{6} (f(x_0) + 4 f(\frac{x_0+x_1}{2}) + f(x_1))$
	
	\item Zusammengesetzt \\
	\begin{equation*}
		S^{(n)}(f) = \frac{h}{6} \bigg(f(x_0) + 2 \sum_{k=1}^{n-1}f(x_k) + 4 \sum_{k=1}^nf\big(\frac{x_{k-1}+x_k}{2} \big) + f(x_n) \bigg)
	\end{equation*}
	
	\item Fehler \\
	$e_v = \bigg|\int_a^b (f(x)-p(x)) dx \bigg| \leq \frac{M_4(b-a)^5}{2880}$
\end{itemize}

\subsection{Newton-Cotes}
\begin{itemize}

	\item Allgemeine Formel \\
	\begin{equation*}
		P(f) = (b-a) \sum_{i=0}^n \alpha_i f(x_i)
	\end{equation*}
	
	\item Tableau \\
	\begin{displaymath}
		\begin{array}{l | lllll | l}
			n & a_i & & & & & name \\
			\hline
			1 & \frac{1}{2} & \frac{1}{2} & & & & Trapez \\
			2 & \frac{1}{6} & \frac{4}{6} & \frac{1}{6} & & & Simpson \\
			3 & \frac{1}{8} & \frac{3}{8} & \frac{3}{8} & \frac{1}{8} & & \frac{3}{8}-Regel \\
			4 & \frac{7}{90} & \frac{32}{90} & \frac{12}{90} & \frac{32}{90} & \frac{7}{90} & Milne/Boole \\
		\end{array}
	\end{displaymath}
	
\end{itemize}

\subsection{Gaus-Legendre Stützstellen}
\begin{itemize}

	\item Formeln \\
	$x_k = \frac{b-a}{2} \widetilde{x}_k + \frac{b+a}{2}$ \\
	$\alpha_k = \frac{b-a}{2} \widetilde{\alpha}_k$
	
	\item Werte für $\widetilde{x}_k$ \\
	\begin{displaymath}
		\begin{array}{l | llll}
			n & \widetilde{x}_k \\
			\hline
			0 & 0 \\
			1 & - \sqrt{\frac{1}{3}} & \sqrt{\frac{1}{3}} \\
			2 & - \sqrt{\frac{3}{5}} & 0 & \sqrt{\frac{3}{5}} \\
			3 & 
				- \sqrt{\frac{3}{7} + \frac{2}{7}\sqrt{\frac{6}{5}}} & 
				- \sqrt{\frac{3}{7} - \frac{2}{7}\sqrt{\frac{6}{5}}} & 
				\sqrt{\frac{3}{7} - \frac{2}{7}\sqrt{\frac{6}{5}}} & 
				\sqrt{\frac{3}{7} + \frac{2}{7}\sqrt{\frac{6}{5}}} \\
		\end{array}
	\end{displaymath}
	
	\item Werte für $\widetilde{\alpha}_k$ \\
	\begin{displaymath}
		\begin{array}{l | llll}
			n & \widetilde{\alpha}_k \\
			\hline
			0 & 2 \\
			1 & 1 & 1 \\
			2 & \frac{5}{9} & \frac{8}{9} & \frac{5}{9} \\
			3 & 
				\frac{18 - \sqrt{30}}{36} & 
				\frac{18 + \sqrt{30}}{36} & 
				\frac{18 + \sqrt{30}}{36} & 
				\frac{18 - \sqrt{30}}{36} \\
		\end{array}
	\end{displaymath}
		
\end{itemize}

\subsection{Romberg}
\begin{itemize}
	
	\item Allgemeine Formel \\
	\begin{displaymath}
		T_h^{(k)} = \frac{2^{2k} T_{h/2}^{(k-1)} - T_h^{(k-1)}}{2^{2k}-1}
	\end{displaymath}
	
	\item Tableau \\
	\begin{displaymath}
		\begin{array}{l | llll}
			h & T_h^{(0)} \\
			\frac{h}{2} & T_{h/2}^{(0)} & T_{h}^{(1)} \\
			\frac{h}{4} & T_{h/4}^{(0)} & T_{h/2}^{(1)} & T_{h}^{(2)} \\
			\frac{h}{8} & T_{h/8}^{(0)} & T_{h/4}^{(1)} & T_{h/2}^{(2)} & T_{h}^{(3)} \\
		\end{array}
	\end{displaymath}
	
	\item Diagonale Ausgerechnet \\
	\begin{displaymath}
		\begin{array}{cccc}
			0 & 1 & 2 & 3 \\
			\hline
			T_h^{(0)} & 
			\frac{4T_{h/2}^{(0)} - T_h^{(0)}}{3} &
			\frac{16T_{h/2}^{(1)} - T_h^{(1)}}{15} &
			\frac{64T_{h/2}^{(2)} - T_h^{(2)}}{63}
		\end{array}
	\end{displaymath}
	
	\item Bedingung für Termination \\
	\begin{equation*}
		\frac{|T_{h}^{(k)}(f) - T_{h}^{(k-1)}(f)|}{|T_{h}^{(k)}(f)|} \leq TOL
	\end{equation*}
		
\end{itemize}

\subsection{Adaptiv}
\begin{enumerate}
	
	\item Starte mit $I_0 = [a, b]$ und TOL: \\
	\begin{itemize}
		\item $S_{[a,b]}^{(0)}(f)$
		\item $S_{[a,b]}^{(1)}(f) = S_{[a,\frac{a+b}{2}]}^{(0)}(f) + S_{[\frac{a+b}{2},b]}^{(0)}(f)$
	\end{itemize}
	
	\item Wenn $|S_{[a,b]}^{(0)}(f) - S_{[a,b]}^{(1)}(f)| > (2^q-1)\cdot TOL$ dann: \\
	
	\begin{enumerate}
		
		\item Halbiere Intervall I in $[a_1, b_1]$ und $[a_2, b_2]$	
		
		\item Wiederhole den Algorithmus für jedes Teilintervall
		
	\end{enumerate}
	
	\item Fehler
	\begin{equation*}
		E_h(f)= I(f) - S_h(f) \approx \frac{2^q}{2^q-1}(S_{h/2}(f)-S_h(f))
	\end{equation*}
	\begin{itemize}
		\item Simpson: $q = 4$
		\item Trapez: $q = 2$
	\end{itemize}
	
\end{enumerate}