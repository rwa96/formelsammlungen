\section{Lineare Ausgleichsrechnung}

\subsection{Normalengleichung}
\subsubsection*{Allgemeine Formel}
\begin{itemize}

	\item $f(x) = \alpha_1f_1(x) + \cdots + \alpha_nf_n(x)$
	
	\item Ausgleichsproblem
	\begin{displaymath}
		A = 
		\begin{pmatrix}
			f_1(x_1) & \cdots & f_n(x_1) \\
			\vdots & \ddots & \vdots \\
			f_1(x_m) & \cdots & f_n(x_m)
		\end{pmatrix}
		, \alpha =
		\begin{pmatrix}
			\alpha_1 \\ \vdots \\ \alpha_n
		\end{pmatrix}
		, b = 
		\begin{pmatrix}
			b_1 \\ \vdots \\ b_m
		\end{pmatrix}
	\end{displaymath}
	\begin{equation*}
		A^TA\alpha = A^Tb
	\end{equation*}

\end{itemize}

\subsection{Householder}
\begin{itemize}
	
	\item Approximierende Funktion: $f(x) = \alpha_1 + \alpha_2 \cdot x$
	
	\item Ausgleichsproblem: $A\alpha = f(\vec{x}) \Rightarrow R\alpha = Q^T \cdot f(\vec{x})$
	
	\item Wende Householder auf Ausgleichsproblem an
	
	\item Residuum (Näherungsfehler)
	\begin{displaymath}
		Q^T\vec{b} = 
		\begin{pmatrix}
			\vdots \\
			res
		\end{pmatrix}
	\end{displaymath}

\end{itemize}