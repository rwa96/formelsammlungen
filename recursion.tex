\section{Rekursion}

\subsection{Primitive Rekursion}
Kann in endlichen Schritten gelöst werden: \\
\textbf{if, for, listen, etc...}

\subsection{Allgemeine Rekursion}
Nicht ohne Fixpunktoperator möglich

\subsubsection{Fixpunktoperator (Y-Combinator)}
\begin{itemize}
    \item \textbf{Fixpunkt:} f = \lambda x.if (< x 2) 1 (+ (f (- x 1)) (f (- x 2)))
    \item \textbf{Rekursion:} F = \lambda fx.if (< x 2) 1 (+ (f (- x 1)) (f (- x 2)))
    \item f = F f \textit{(f ist ein Fixpunkt von F)}
\end{itemize}

Für ein beliebiges \textbf{F} ist \textbf{YF} ein Fixpunkt:
Y = \lambda g.(\lambda x.g(xx)) (\lambda x.g(xx))

\begin{enumerate}
    \item YF = \lambda (g.(\lambda x.g(xx))(\lambda x.g(xx))) F
    \item (\lambda x.F(xx))(\lambda x.F(xx))
    \item F((\lambda x.F(xx)) (\lambda x.F(xx)))
    \item F((\lambda g.(\lambda x.g(xx)) (\lambda x.g(xx))) F)
\end{enumerate}
