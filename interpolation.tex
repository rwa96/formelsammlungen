\section{Interpolation}

\subsection{Vandermonde}

\begin{itemize}

	\item Allgemeine Formel \\
	\begin{equation*}
		p(x) = c_{0} + c_{1}x + c_{2}x^2 + c_{n-1}x^{n-1}
	\end{equation*}
	\begin{equation*}
		A\vec{x} = \vec{y}
	\end{equation*}
	
	\item Lösungsmatrix \\
	\begin{displaymath}
		A = 
		\begin{pmatrix}
			1 		& x_{1} 	& x_{1}^2 	& \cdots	& x_{1}^{n-1} 	\\
			\vdots	& \vdots	& \vdots	& \vdots	&\vdots			\\
			1 		& x_{n} 	& x_{n}^2 	& \cdots	& x_{n}^{n-1} 	\\
		\end{pmatrix}
	\end{displaymath}
	
	\item Hornerschema \\
	\begin{equation*}
		p(x) = (...((c_{n} x + c_{n-1}) x + c_{n-2}) x + \cdots + c_{1}) x + c_{0}
	\end{equation*}
	
	\item Aufwand: $O(n^3)$

\end{itemize}

\subsection{Lagrage}

\begin{itemize}
	
	\item Einzelpolynom \\
	\begin{displaymath}
		L_{j} = \prod_{k=0, k\neq j} \frac{x - x_{k}}{x_{j} - x_{k}}
		\implies
		\frac{(x - x_{2}) (x - x_{3}) \cdots (x - x_{n})}{(x_{1} - x_{2}) \cdots (x_{1} - x_{n})}
	\end{displaymath}
	
	\item Gesamtpolynom \\
	\begin{equation*}
		p(x) = \sum_{i=0}^{n} y_{i} L_{i}
	\end{equation*}
	
	\item Aufwand: $O(n^{2})$
	
\end{itemize}

\subsection{Newton}

\begin{itemize}
	
	\item Allgemeine Formel \\
	\begin{equation*}
		p_n(x) = y_0 + y_{10} (x - x_0) + y_{210} (x - x_0)(x - x_1) + \cdots + y_{n\cdots 3210} (x - x_0)(x - x_1) \cdots (x - x_n)
	\end{equation*}
	
	\item Dividierte Differenzen (Diagonale y-Werte sind b-Werte)\\
	\begin{displaymath}
		\begin{array}{l | l l l l}
			x_0		& y_0 		\\
			x_1		& y_1		& y_{10}	\\
			x_2		& y_2		& y_{21}	& y_{210}	\\
			x_3		& y_3		& y_{32}	& y_{321}	& y_{3210} \\
			\vdots	& \vdots	& \vdots	& \vdots	& \ddots \\
		\end{array}
	\end{displaymath}
	\begin{displaymath}
		y_{k\cdots l} = \frac{y_{k\cdots (l+1)} - y_{(k-1)\cdots l}}{x_k - x_l}
		\implies
		y_{321} = \frac{y_{32} - y_{21}}{x_3 - x_1}
	\end{displaymath}
	
	\item Aufwand: $O(n^2)$ (mit besserer Konstante)
	
\end{itemize}

\subsection{Chebyshev Stützstellen}

\begin{equation*}
	x_j = \frac{a+b}{2} + \frac{b-a}{2} \cos\bigg(\frac{2j + 1}{2n + 2} \pi\bigg)
\end{equation*}

\subsection{Kubische Splines}

\begin{itemize}
	
	\item Allgemeine Formel \\
	\begin{equation*}
		s_i(x) = a_i + b_i (x - x_i) + c_i (x - x_i)^2 + d_i (x - x_i)^3
	\end{equation*}
	
	\item Formeln für die Parameter \\
	\begin{enumerate}
		\item $h_i = x_{i+1} - x_i$
		\item $a_i = y_i$
		\item $A\vec{c}=\vec{r} \implies c_i$
		\begin{displaymath}
			\begin{pmatrix}
				h_0		& 2(h_0 + h_1)	& h_1	& 0 		& \cdots	& 0 \\
				0		& h_1	& 2(h_1 + h_2)	& h_2		& \cdots	& \vdots \\
				\vdots	& \vdots	& \vdots	& \vdots	& \vdots	& \vdots \\
				0		& \cdots	& \cdots	& h_{n-2}	& 2(h_{n-2} + h_{n-1}	& h_{n-1} \\
			\end{pmatrix}
			\vec{c} = 
			\begin{pmatrix}
				3\frac{y_2-y_1}{h_1} - 3\frac{y_1-y_0}{h_0} \\
				3\frac{y_3-y_2}{h_2} - 3\frac{y_2-y_1}{h_1} \\
				\vdots \\
				3\frac{y_n-y_{n-1}}{h_{n-1}} - 3\frac{y_{n-1}-y_{n-2}}{h_{n-2}} \\
			\end{pmatrix}
		\end{displaymath}
		\item $b_i = \frac{y_{i+1} - y_i}{h_i} - \frac{h_i}{3} (c_{i+1} + 2c_i)$
		\item $d_i = \frac{c_{i+1} - c_i}{3 h_i}$
	\end{enumerate}
	
	\item Desweiteren gilt \\
		$2 c_i + 6 d_i h_i = 2 c_{i+1}$ \\
		$b_i + 2 c_i h_i + 3 d_i h_i^2 = b_{i+1}$
	
	\item Natürliche Splines (Erste und letzte Spalte von A fallen weg)
	(Spline ist eine Gerade außerhalb des Intervalls) \\
	$c_0 = c_n = 0$
	
	\item Not-a-knot Splines (Erste u. letzte Spalte von A fallen weg)
	(Stetigkeit der 3. Ableitung für $x_1$ und $x_{n-1}$) \\
	$6d_0 = 6d_1 \wedge 6d_{n-2} = 6d_{n-1}$ 
	\begin{displaymath}
		\implies \vec{r} =
		\begin{pmatrix}
			\frac{h_1}{h_0 + h_1} r_1 \\
			r_2 \\
			\vdots \\
			r_{n-2} \\
			\frac{h_{n-2}}{h_{n-2}+h_{n-1}} r_{n-1} \\
		\end{pmatrix}
	\end{displaymath}
	
\end{itemize}
