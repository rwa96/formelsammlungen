\section{Lineare Gleichungssysteme}

\subsection{QR-Zerlegung: Householder}

\subsubsection*{Eigenschaften der Q-Matrix}
\begin{itemize}

	\item $Q^{-1} = Q^T$
	
	\item $||Q||_2 = 1$
	
	\item $Q^TQ = QQ^T = I$
	
	\item für jede reguläre Matrix existiert eine QR-Zerlegung
	\begin{itemize}
	
		\item A ist regulär wenn es eine inverse gibt
		
		\item regulär wenn rank(A) = n	
	
	\end{itemize}

\end{itemize}

\subsubsection*{Verfahren}
\begin{equation*}
	A = Q R \Rightarrow R \cdot \vec{x} = Q^T \cdot \vec{b}
\end{equation*}
Berechnung von $Q^T \cdot \vec{b}$ und R
\begin{displaymath}
	A =
	\begin{pmatrix}
		\vec{a}_1 & \vec{a}_2 & \cdots
	\end{pmatrix}
\end{displaymath}
\begin{enumerate}

	\item Householdervektor
	\begin{displaymath}
		\vec{v}_1 = \vec{a}_1 + sign(\vec{a}_{11}) \cdot ||\vec{a}_1||_2 \cdot
		\begin{pmatrix}
			1 \\ 0 \\ \vdots \\ 0
		\end{pmatrix}
	\end{displaymath}
	
	\item Spiegelung
	\begin{equation*}
		Q_1a_i = a_i - \frac{2}{v_1^Tv_1}v_1(v_1^T a_i)
	\end{equation*}
	\begin{equation*}
		Q_1b = b - \frac{2}{v_1^Tv_1}v_1(v_1^T b)
	\end{equation*}
	
	\item Iteration
	\begin{displaymath}
		A_2 = Q_2 =
		\begin{pmatrix}
			Q_1a_1 & Q_1a_2 & \cdots & Q_1a_n
		\end{pmatrix}
	\end{displaymath}
	\begin{displaymath}
		Q_2 =
		\begin{pmatrix}
			1 & 0 & \cdots \\
			0 & \widetilde{Q}_2 & \\
			\vdots & & \\
		\end{pmatrix}
	\end{displaymath}
	\begin{displaymath}
		\vec{b}_2 =
		\begin{pmatrix}
			1 \\
			\widetilde{b}_2 \\
			\vdots \\
		\end{pmatrix}
	\end{displaymath}
	
	\item Verwende $\widetilde{Q}_2$ und $\widetilde{b}_2$ und wiederhole schritte
	\begin{equation*}
		\vec{a}_i := Q_1\widetilde{a}_{i+1}
	\end{equation*}
	\begin{equation*}
		\vec{b}_i := Q_1\widetilde{b}_{i+1}
	\end{equation*}
	
	\item Zusammensetzen
	\begin{displaymath}
		R =
		\begin{pmatrix}
			Q_1 & & \\
			& \widetilde{Q}_2 & \\
			& & \ddots
		\end{pmatrix}
	\end{displaymath}
	\begin{displaymath}
		Q^T \vec{b} =
		\begin{pmatrix}
			b_1 \\
			\widetilde{b}_2 \\
			\vdots
		\end{pmatrix}
	\end{displaymath}

\end{enumerate}

\subsection{Lineare Ausgleichsrechnung: Householder}
\begin{itemize}
	
	\item Approximierende Funktion: $f(x) = \alpha_1 + \alpha_2 \cdot x$ \\
	
	\item Ausgleichsproblem: $A\alpha = f(\vec{x}) \Rightarrow R\alpha = Q^T \cdot f(\vec{x})$
	
	\item Wende Householder auf Ausgleichsproblem an
	
	\item Residuum (Näherungsfehler)
	\begin{displaymath}
		Q^T\vec{b} = 
		\begin{pmatrix}
			\vdots \\
			res
		\end{pmatrix}
	\end{displaymath}

\end{itemize}
