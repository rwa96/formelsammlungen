\section{Differenzieren von Funktionen des Typs $\mathbb{R}^n \rightarrow \mathbb{R}^m$}
\begin{displaymath}
	\begin{pmatrix}
		x_1 \\
		\vdots \\
		x_n
	\end{pmatrix}
	\rightarrow
	\begin{pmatrix}
		f_1(x_1,\cdots , x_n) \\
		\vdots \\
		f_m(x_1,\cdots , x_n)
	\end{pmatrix}
\end{displaymath}

\subsection{Ableitung}
\begin{displaymath}
	f'
	\begin{pmatrix}
		x_1 \\
		\vdots \\
		x_n \\
	\end{pmatrix}
	=
	\begin{pmatrix}
		\frac{\delta f_1}{\delta x_1} & \cdots & \frac{\delta f_1}{x_n} \\
		& \vdots & \\
		\frac{\delta f_m}{\delta x_1} & \cdots & \frac{\delta f_m}{x_n} \\
	\end{pmatrix}
\end{displaymath}

\subsection{Linearisierung}
\subsubsection{Normale Funktionen}
\begin{displaymath}
	f
	\begin{pmatrix}
		x_1 + \Delta x_1 \\
		\vdots \\
		x_n + \Delta x_n
	\end{pmatrix}
	\rightarrow
	\begin{pmatrix}
		f_1(x_1,\cdots , x_n) \\
		\vdots \\
		f_m(x_1,\cdots , x_n)
	\end{pmatrix}
	+
	\begin{pmatrix}
		\frac{\delta f_1}{\delta x_1} & \cdots & \frac{\delta f_1}{\delta x_n} \\
		& \vdots & \\
		\frac{\delta f_m}{\delta x_1} & \cdots & \frac{\delta f_m}{\delta x_n}
	\end{pmatrix}
	\cdot
	\begin{pmatrix}
		\Delta x_1 \\
		\vdots \\
		\Delta x_n
	\end{pmatrix}
\end{displaymath}

\subsubsection{Verkettete Funktionen}
\begin{enumerate}
    \item Ansatz: \textbf{Vereinfachen} \\
    $(g \circ f)' = g'(f) = g'_f$

    \item Ansatz: \textbf{Kettenregel} \\
    $(g \circ f)' = (g' \circ f) \cdot f'$
\end{enumerate}

\subsection{Differenzialgleichung 1. Ordnung}
\begin{equation*}
    \varphi(x)' = f(x, \varphi(x)) \Rightarrow \varphi(x) = \int_{x_0}^x \varphi(t)' dt + \varphi(x_0)
\end{equation*}
\begin{itemize}
    \item Sonderfall: $f(x)$ unabhängig von  $\varphi(x)$ \\
    $\Rightarrow$ Lösung

    \item Allgemein: $\varphi(x)' = f(x, \varphi(x))$ \\
    $\Rightarrow$ Näherung durch $\varphi(x) = (x - x_0) \cdot f(x_0, \varphi(x_0)) + \varphi(x_0)$ \\
    \textit{für kleine} $|x - x_0|$
\end{itemize}

\subsection{System von Differenzialgleichungen 1. Ordnung}
\begin{equation*}
    y_1' = f(x,y_1, \cdots y_n), \cdots y_n' = f(x,y_1, \cdots y_n)
\end{equation*}
\begin{equation*}
    \Rightarrow \varphi(x) = \varphi(x_0) + f(x_0, \varphi(x_0)) \cdot (x - x_0)
\end{equation*}

\subsubsection{Einzelteile der Gleichung}
\begin{displaymath}
    \varphi(x) =
    \begin{pmatrix}
        \varphi(x)_1 \\
        \varphi(x)_2 \\
        \vdots \\
        \varphi(x)_n
    \end{pmatrix}
\end{displaymath}

\begin{displaymath}
    \varphi(x)' =
    \begin{pmatrix}
        \varphi(x)'_1 \\
        \varphi(x)'_2 \\
        \vdots \\
        \varphi(x)'_n
    \end{pmatrix}
\end{displaymath}

\begin{displaymath}
    f
    \begin{pmatrix}
        x,
        \begin{pmatrix}
            y_1 \\ \vdots \\ y_n
        \end{pmatrix}
    \end{pmatrix}
    =
    \begin{pmatrix}
        f(x, y_1, \cdots y_n)_1 \\
        \vdots \\
        f(x, y_1, \cdots y_n)_n
    \end{pmatrix}
    =
    \begin{pmatrix}
        y_1' \\
        \vdots \\
        y_n'
    \end{pmatrix}
\end{displaymath}

\subsection{Lippschitzbedingung}
\begin{equation*}
    |f(x, y) - f(x, \tilde{y})| \leq |y - \tilde{y}| \cdot L
\end{equation*}

\subsection{Eindeutigkeitssatz}
Es seien $\varphi$ und $\psi$ zwei Lösungen der DGL $\varphi, \psi: \mathbb{R} \rightarrow \mathbb{R}^n$. \\
Gilt $\varphi(x_0) = \psi(x_0)$ für ein $x_0 \in \mathbb{R}$, \\
so gilt $\varphi(x) = \varphi(\psi)$ für alle $x$

\subsection{Differenzialgleichung n-ter Ordnung}
\begin{equation*}
    f(x, y, y^{(1)}, y^{(2)}, \cdots y^{(n-1)}) = y^{(n)}
    \Rightarrow
    \varphi^{(n)}(x) = f(x, \varphi(x), \varphi^{(1)}(x), \varphi^{(2)}(x), \cdots \varphi^{(n-1)}(x))
\end{equation*}
